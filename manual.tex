%% LyX 1.6.4 created this file.  For more info, see http://www.lyx.org/.
%% Do not edit unless you really know what you are doing.
\documentclass[english]{article}
\usepackage[T1]{fontenc}
\usepackage[latin9]{inputenc}
\usepackage{amstext}
\usepackage{amssymb}
\usepackage{babel}

\begin{document}

\title{Input Prepartion Guide for the Program ppp.x}

\maketitle
All the input data is written in the free format to minimize errors.
Therefore, before each important input card, there is a compulsory
comment line to make the input self explanatory. It is irrelevant
as to what is written in the comment lines, but, by writing something
meaningful, one can keep the input process transparent. We do require
that all the ASCII input cards (not the ones in the comment lines)
should be in uppercase letters. To distinguish comment lines from
the input, in this guide we precede each comment line with the character
'\#', although in actual input files, that is not necessary to use.
We encourage the user to compare the explanation provided here with
the actual input files available in the 'examples' subdirectory. Next,
we explain the preparation of input files card by card. 
\begin{enumerate}
\item The input file starts with a title. Thus, the first line is for a
title of the calculations. A user can give a relevant title so as
to provide an idea about the calculation or system at the first glance.
\medskip{}
\\
\# Title 
\item As the purpose of our code is to perform calculations using various
semi-empirical model Hamiltonians, the first input card is an ASCII
card describing the model Hamiltonian to be used for the calculations.
Options are: P-P-P, Hubbard, extended Hubbard, and H�ckel.\medskip{}
\\
 (a) For a system using P-P-P model, the input card may read:\medskip{}
\\
 \# The Hamiltonian to be used\\
PPP \medskip{}
\\
(b) For a system using Hubbard model, the input card may read:\medskip{}
\\
 \# The Hamiltonian to be used\\
HUBBARD\medskip{}
\\
(c) For a system using extended Hubbard model, the input card may
read:\medskip{}
\\
 \# The Hamiltonian to be used\\
EXTHUB \medskip{}
\\
(d) For a system using H�ckel model, the input card may read:\medskip{}
\\
 \# The Hamiltonian to be used\\
HUECKEL
\item If P-P-P model Hamiltonian is to be used, one can use various parametrization.
Thus, our second card is also an ASCII card, stating which parametrization
is to use. Options are: Ohno, Mataga-Nishimoto, and exponential parametrization.\medskip{}
\\
 (a) For Ohno parametrization, the input card may read:\medskip{}
\\
 \# Parameterization for the P-P-P Hamiltonian\\
OHNO\\
\medskip{}
\\
(b) For Mataga-Nishimoto parametrization, the input card may read:\medskip{}
\\
 \# Parametrization for the P-P-P Hamiltonian\\
MATNIS\\
\medskip{}
\\
(c) For exponential parametrization, the input card may read:\medskip{}
\\
 \# Parametrization for the P-P-P Hamiltonian\\
EXP\\

\item The long-range Coulomb potentials in case of Ohno parametrization
can be parametrized using different sets of parameters, such as standard
parameters or screened parameters. If STANDARD is given as input card,
the program automatically takes $U$=11.13d0, $r_{0}$=1.2785884d0,
and $\kappa_{i,j}$=1.d0, while in case of input card SCREENED, it
takes $U$, $r_{0}$, and $\kappa_{i,j}$ as 8.d0, 1.2785884d0, and
2.d0, respectively. However, one can also directly give the parameters.
If the user opts for Hubbard or extended Hubbard model, then parameters
should be provided explicitly.\\
If Ohno parametrization is set as an option then the next card
will be an ASCII, mentioning the parameters to use.\medskip{}
\\
 (a) For standard parameters, the input card may read:\medskip{}
\\
 \# Parameters for the Ohno parametrization in the P-P-P Hamiltonian\\
STANDARD\medskip{}
\\
 (b) For screened parameters, the input card may read:\medskip{}
\\
 \# Parameters for the Ohno parametrization in the P-P-P Hamiltonian\\
SCREENED\medskip{}
\\
 (c) For other parameters, the input card may read:\medskip{}
\\
 \# Parameters for the Ohno parametrization in the P-P-P Hamiltonian\\
PARA\\
11.d0,1.28d0,1.d0\\
\medskip{}
\\
If the user wants to use other than Ohno parametrization such as
Mataga-Nishimoto or exponential then one needs to just provide the
values of $U$ and $r_{0}$. For example,\\
\medskip{}
\\
 \# Parameters for the Mataga-Nishimoto or exponential parametrization
in the P-P-P Hamiltonian\\
11.d0, 1.28d0\\
\medskip{}
\\
If one is using Hubbard model, the value for $U$ should be given
directly.\medskip{}
\\
 11.d0\medskip{}
\\
 In case of extended Hubbard model calculations, the value for
$U$ and $V$ (nearest neighbours potential) should also be given
directly.\medskip{}
\\
11.d0,3.d0\\
\medskip{}

\item Next card will read the electric charge (ionicity) on the system.
The magnitude of the nuclear charges have been defined as unity ($=1$).
For the neutral system the ionicity will be zero, and so on. \medskip{}
\\
For example, in case of $trans$-polyacetylene, the card will read
as follows\medskip{}
\\
\#Charge on the system\\
0\\
\medskip{}
\\
In order to handle a singly ionized cation the input is:\medskip{}
\\
\#Charge on the system\\
1\\
\medskip{}
\\
while for a singly ionized anion, the input will be:\medskip{}
\\
\#Charge on the system\\
-1
\item As our code has been framed to handle mainly molecules and polymers,
the next input card will read the total number of atoms in the unit
cell, after the comment line. \medskip{}
\\
 For example, if we are handling a system having two number of
atoms in the unit cell, then the card will read \medskip{}
\\
\# Total number of atoms in the unit cell\\
2
\item Next card deals with the Cartesian coordinates of the different atoms
in the unit cell, to be provided in the units of $\textrm{\AA}$.
We have two options to facilitate this: (a) Automatic generation through
various keywords, (b) by explicitly providing coordinates of each
atom. First we will describe the keywords involved in the automatic
generation of coordinates followed by one example illustrating their
use. \medskip{}
\\
 \# Give the coordinates of the atoms involved in the system\\
\medskip{}
\\
 (a) For example, for the benzene ring oriented in parallel direction
to the conventional orientation, in the $xy$-plane, the input card
may read:\medskip{}
\\
 BEN-XY\\
\medskip{}
In case of ring lying in $yz$- or $zx$-plane, one should give card
as 'BEN-YZ' or 'BEN-ZX', respectively.\medskip{}
\\
 The next card will specify the center of the benzene ring. For
example input, \medskip{}
\\
-3.2725,0.575907,0.0 \\
implies that the phenyl ring is centered at the point (-3.2725,0.575907,0.0),
in Cartesian coordinates .\\
\medskip{}
\\
The next card must specify the bond length of the hexagon constituting
the phenyl ring and whether we need to rotate this ring with respect
to some axis. For example input, \\
\medskip{}
\\
1.4,0\\
implies that the bond length is 1.4 \AA~and no rotations need
to be performed on the ring. \\
\medskip{}
\\
If instead this input is like:\\
 \medskip{}
\\
1.4,2 \\
3,-90.d0\\
 2,30.d0\\
which implies that the C--C bond length in the ring is 1.4 \AA,
which will undergo two subsequent rotations. First rotation is a clockwise
one by $90^{\circ}$ about the $z$-axis, followed by second counter
clockwise rotation by $30^{\circ}$ about the $y$-axis. For example,
in an entry of the type 3, -90.d0, first number denotes the axis of
the rotation, while the second number denotes the angle of the rotation.
\\
\medskip{}
\\
 (b) For a benzene ring oriented in perpendicular direction to
the conventional orientation, in the $xy$-plane, the keyword will
be:\medskip{}
\\
BENP-XY\medskip{}
\\
and rest of the input will be identical to the case of BEN-XY,
discussed above. \medskip{}
\\
(c) For generating a bond (a unit consisting of two atoms separated
by a distance) whose location is specified by the coordinates of the
bond center, the input card is:\medskip{}
\\
BOND\medskip{}
\\
 Next to this card, the user should specify origin, bond length
and number of rotations to be performed on the bond because by default
the bond is assumed to be along the $x$-axis. Rotations, if desired,
are performed keeping the center fixed. Positive angles imply counter-clockwise
rotations, while the negative angles denote clockwise rotations. For
example input, \medskip{}
\\
1.0,0.0,0.0\\
1.33,1\\
3,-60.d0\\
will initially generate a bond centered at the Cartesian coordinates
(1.0, 0.0, 0.0) of length 1.33 \AA along the $x$-axis, then subsequently
it will be rotated with respect to the $z$-axis by $60^{\circ}$
angle (clockwise).\\
\medskip{}
\\
(d) For generating a bond whose first atom lies at the user specified
location (as against the center lying at the user specified location
in case of the BOND keyword), an alternative input card called LINE
is provided:\medskip{}
\\
LINE\medskip{}
\\
 rest of the inputs are identical to those needed with the keyword
BOND.\\
\medskip{}
\\
(e) Keyword ATOM is provided to generate a single atomic site located
at the user specified coordinates. The input:\medskip{}
\\
ATOM\\
 0.0,0.0,0.0\\
will generate an atomic site located at the origin.\\
\medskip{}
\\
(f) A keyword C60 is provided to generate the atomic coordinates
of a bond alternating fullerene C$_{60}$. For example the input,\medskip{}
\\
C60\\
1.45,1.35\\
1.0,1.0,1.0\\
will generate the coordinated of 60 carbon atoms constituting a
fullerene molecule center located at point (1.0,1.0,1.0), with single
and double bond lengths, 1.45 \AA and 1.35 \AA respectively. \\
\medskip{}
\\
(g) If one does not want to use the above mentioned keywords and
straight away wants to provide Cartesian coordinates of all the atoms
located in the system, keyword COORD should be used. For example input,\medskip{}
\\
COORD\\
0.0,0.0,0.0\\
1.4,0.0,0.0\\
0.0,1.4,0.0\\
1.4,1.4,0.0\\
will generate a system with four atoms located at positions (0.0,0.0,0.0),
(1.4,0.0,0.0), (0.0,1.4,0.0), and (1.4,1.4,0.0).\\
\medskip{}
\\
(h) In all the cases the end of the inputs for the atomic coordinates
generation is signalled by the keyword ENDA. \medskip{}
\\
ENDA\\
\\
\medskip{}
\\
Next we provide couple of examples of coordinate inputs which will
illustrate the use of these keywords. The first example specifies
the generation of the atomic coordinates of a unit cell of the polymer
phenyl disubstituted polyacetylene (PDPA), considered in several of
our papers (\emph{e.g.}, P. Sony and A. Shukla, Phys. Rev. B 71 (2005)
165204.). This input illustrates the use of the some of the keywords
discussed above.\\
 \\
BEN-XY\\
-0.577485,3.289479,0.0\\
1.4,2\\
3,-90.d0\\
2,30.d0\\
BOND\\
0.0,0.0,0.0\\
1.35,1\\
3,-31.18125\\
BEN-XY \\
0.577485,-3.289479,0.0\\
1.4,2\\
3,-90.d0\\
2,30.d0\\
ENDA\\
\\
The next example illustrates a case of a benzene ring centered
at (-2.4248,1.4000,0.0000), where atomic coordinates are provided
directly.\\
\\
COORD\\
2.42480 -2.80000 0.00000\\
-3.63720 -2.10000 0.00000\\
-3.63720 -0.70000 0.00000\\
-2.42480 0.00000 0.00000\\
-1.21240 -0.70000 0.00000\\
-1.21240 -2.10000 0.00000\\
ENDA\\

\item The next card is again an ASCII card, which will contain information
regarding the type of calculations one wants to perform. \medskip{}
\\
\# Type of Calculations\medskip{}
\\
(a) For example, if one wants to do restricted Hartree-Fock calculations,
then the input card should be:\medskip{}
\\
RHF\\
\medskip{}
\\
(b) In case of unrestricted Hartree-Fock calculations, the input
card will be:\medskip{}
\\
UHF\\
\medskip{}
\\
For the UHF card above additional input specifying the number of
up spin (nalpha) and down spins (nbeta) electrons is essential\\
\medskip{}
\\
\#Number of up and down spin electrons\\
7, 6\\
The input above implies that the system contains 13 electrons in
all, 7 of which are of up spin and remaining 6 of down spin.\\
\medskip{}
\\
(c) If followed by RHF calculations one wants to perform a single
CI calculation, the corresponding input card is:\medskip{}
\\
SCI\\
\medskip{}
\\
The following example illustrates a situation where both RHF and
SCI calculations are performed.\\
\\
\#TYPE OF CALCULATION\\
RHF\\
\#SCI CALCULATIONS TO BE PERFORMED\\
SCI \\
\medskip{}
\\
(d) The input card OPTICS should be used when one wants to perform
calculations of linear optical absorption on the given system. At
present this card cannot be used with a UHF calculation. Keyword OPTICS
is to be followed by $d\omega$,$\omega_{min}$,$\omega_{max}$,$\Gamma$,
$scale$, where $d\omega$ , represents the frequency step, $\omega_{min}/\omega_{max}$
represent minimum/maximum range of frequencies for which the spectrum
is to be computed, while $\Gamma$ is the line width of the excited
states, and $scale$ is a number which can be used to convert the
energy unit from the default eVs to a unit of user's choice. If the
energy unit of eVs is to be used then this number can be given to
be 1.0, 0.0, or any negative number. For single particle calculations
such as H�ckel model or RHF one needs to specify that from how many
reference states we want to compute the linear absorption spectrum.
Thus, one can compute both the ground state, as well as excited state
absorptions. For excited states absorption, at present we are restricted
to only those excited states which can be obtained by single excitations
from the ground state. The input format is:\medskip{}
\\
\#OPTICS TO BE PERFORMED\\
OPTICS\\
\#Read the number of dipole moment components needed and line-width
parameters\\
3 \# dimensions\\
1,2,3 \# directions\\
0.1, 0.0, 10.0, 0.01, 1.0 \# $\Gamma$, $\omega_{min}$,$\omega_{max}$,
$d\omega$, scale\\
\#From how many states one wants to compute linear absorption,
'0' implies ground state\\
1 \# no. of states from which absorption will be computed\\
0 \# this implies that absorption is computed from the ground state\\
\medskip{}
\\
If the state from which absorption is to be computed is not the
ground state but an excited state (\emph{i.e.}, not '0', but '1'),
then we need to specify the orbitals in which holes and electrons
exists to characterize the excited state in question. The input will
be: \\
\medskip{}
\\
\#From how many states one wants to compute linear absorption,
'0' implies ground state\\
1 \# no. of states from which absorption will be computed\\
1 \# this implies that absorption is computed from the excited
state\\
4, 5 \# excited state has a hole in orbital 4 and an electron in
orbital 5\\
\medskip{}
\\
The above inputs specifies the linear absorption calculations in
which all the three Cartesian components of the transition dipole
will be used and the spectrum will be computed in the range from 0.0
to 10.0 eV in the steps of 0.01 eV with a linewidth of 0.1 eV for
the H�ckel model or RHF calculations. In the first case the absorption
is from the ground state and in the second case it is from an excited
state.\\
\\
If optical absorption spectrum is to be computed for an SCI calculations
then the input will be like: \\
 \\
\#TYPE OF CALCULATION \\
RHF \\
\#SCI CALCULATIONS TO BE PERFORMED \\
SCI \\
\#OPTICS TO BE PERFORMED \\
OPTICS \\
\#GIVE THE NUMBER OF DIPOLE MOMENTS \\
3 \#dimensions\\
1 2 3 \#directions\\
0.1 0.0 20.0 0.01 1.0 \# $\Gamma$, $\omega_{min}$,$\omega_{max}$,
$d\omega$, scale\\
\#Read the number of states and the states from which absorption
is to be computed \\
1 \# no. of states from which absorption will be computed\\
1 \# this implies that absorption is computed from the excited
state, '0' will imply the absorption from the ground state\\
\\
The absorption spectrum in all the cases is written in a file spec001.dat,
which can plotted using graphical programs such as gnuplot or xmgrace.\\
\medskip{}
\\
(e) If one wants to perform nonlinear optical (NLO) susceptibility
calculations within a single particle model such as H�ckel model or
RHF model, the occupied and unoccupied orbital energies, as well as
dipole matrix elements among them are needed. This data in turn can
be provided to a separate computer program meant for calculating various
NLO susceptibilities. By invoking the NLO card, the user can get the
above mentioned data written in a separate file called NLO001.DAT.
We had written a separate program (not provided with this package)
meant for computing these NLO susceptibilities for which file NLO001.DAT
served as an input. The input in such a case will be: \medskip{}
\\
\#NON LINEAR OPTICS TO BE PERFORMED\\
NLO\\
\#Read the components of dipole moments\\
2 \# dimensions\\
1,2 \# directions\\
\medskip{}
\\
In order to reduce the size of dipole matrix elements, one can
delete the outermost occupied and virtual orbitals choosing the card
'ORBDEL'.\\
\medskip{}
\\
\# Delete occupied and valence orbitals\\
ORBDEL\\
1,10 \# delete occupied orbitals from 1 to 10\\
21,30 \# delete virtual orbitals from 21 to 30\\
\medskip{}
\\
As this is an essential card, so if in case no orbital has to be
deleted, then give the input card:\\
\medskip{}
\\
NOORBDEL\\
\medskip{}
\\
(f) Orbital charge density analysis can be done by giving input
cards as:\medskip{}
\\
ORBDEN\\
0,2 \#providing the density flag and no. of sites.\\
7,8 \#site numbers for which the charge density is desired\\
\medskip{}
\\
If '0' is assigned to the density flag, then only the specified
sites will be analyzed. While if it is assigned a value '1', the sites
given along with their periodic copies will be analyzed.\\
\medskip{}
\\
(g) The band structure calculations can be performed using the
H�ckel model Hamiltonian. To opt for the band structure calculations,
the input cards will be:\\
\medskip{}
\\
BAND\\
-1.0,1.0,0.01 \#lower limit and upper limit of the $k$-values
(crystal momentum), and the differential step\\
\medskip{}
\\
(h) If the user wants to perform post HF correlated calculations
such as FCI, QCI, or MRSDCI, etc., using some other computer program,
various files containing the data such as one- and two-electron integrals
will be required. The card CIPREP serves as the flag to the program
to prepare these files, which are written in the binary format. Thus,
the input card will be given as:\\
\medskip{}
\\
CIPREP\\
\medskip{}
\\
In case of UHF calculations one needs to specify first that which
type of orbitals will be used for the correlation treatment. For this
a flag is provided, whose value is assigned '1' for using the alpha-type
orbitals, and '2' for the beta-type orbitals.\\
\medskip{}
\\
1\\
\medskip{}
\\
If RHF calculations are to be performed then the above line should
be skipped. \\
\\
User can also freeze the occupied orbitals and delete the virtual
orbitals, thus, can perform correlated calculations choosing a specified
set of occupied and virtual orbitals. The input for this can be like
this:\\
\medskip{}
\\
12,0 \# no. of occupied orbitals to be frozen (1$^{\text{st}}$
twelve orbitals will be freezed), flag for freezing\\
11,0 \# no. of virtual orbitals to be deleted, flag for deletion\\
23,33 \# from 23$^{\text{rd}}$ to 33$^{\text{rd}}$ orbital are
to be deleted.\\
\medskip{}
\\
If the electric dipole integrals are needed along with the CIPREP
option (possibly with the aim of performing optical properties calculations
at the correlated level), card DIPINT should be used.\\
\medskip{}
\\
\# Dipole moment calculation\\
DIPINT\\
\#How many components of dipole moments\\
2 \# dimensions\\
1,2 \# directions\\
\medskip{}
If no dipole integral calculation is desired, the input should be:\medskip{}
\\
\# Dipole moment calculation\\
NODIPINT\medskip{}
\\
(i) The orbitals can be printed in the ASCII format in the output
file by using the card:\\
\medskip{}
\\
\#To print the orbitals\\
PRORB\\
\medskip{}
\\
However, this card should be used with caution because, for large
systems, the orbital related data can be huge.\\
\medskip{}
\\
(j) In order to calculate dielectric response properties or electro-absorption
spectrum, one needs to solve the HF equation in the presence of a
finite external electric field. The input card for such calculations
will be:\\
\medskip{}
\\
\#Calculations to be performed in the presence of finite electric
field\\
EFIELD\\
0.001 0.0 0.0 \#components of electric field in $x$, $y$, and
$z$-directions, respectively.\\
\medskip{}
\\
The numbers specifying the components of the external electric
field above are in the units of V/\AA.\\
\medskip{}
\\
(k) After specifying all the methods which are to be used during
the calculations, the end of this input is specified by the following
card:\\
\medskip{}
\\
\#End Method Card\\
ENDM
\item Next card provides the total number of unit cells and should be a
positive integer. For example, for generating an oligomer with eight
units (unit cell coordinate input is specified above), the input will
read:\\
\medskip{}
\\
\#Total number of units\\
8\\

\item If the number of units cells specified in the previous card is greater
than one, translational vector needs to be provided as of the type
below:\\
2.4253866 0.0 0.0\\

\item Atomic operations such as atom deletion can also be used while generating
a given molecule. This operation is useful when for a system containing
more than one unit cell, periodic copies lead to more atoms in the
system than what is needed. For example, for oligomers of polymers
such as PPV and polyacenes, this is usually the case.\\
\medskip{}
\\
For the case of naphthalene atom numbers 11 and 12 need to be deleted
(\emph{cf.} example input files acene2\_ciprep\_orbden.dat).\\
\\
\#If any atomic operations (such as atom deletion) need to be performed
\\
DELATOM\\
2 \# number of atoms to be deleted\\
11, 12 \# atom numbers 11 and 12 are deleted\\
\medskip{}
\\
\#If no atom is to be deleted, the input card should be given as:
\\
NODELATOM\\

\item The convergence threshold set by the user to achieve convergence in
total energy in the units of eV, will be given as: \\
\medskip{}
\\
\#Convergence threshold\\
1.D-8\\

\item The maximum number of iterations allowed to achieve convergence can
be given as follows: \medskip{}
\\
\# Maximum iterations allowed\\
100\\

\item Sometimes, because of the oscillatory nature of total energy during
the SCF iterations it becomes difficult to achieve convergence. However,
in those cases one usually utilizes the techniques of Fock matrix
mixing (which is also popularly called damping). This can be invoked
by the keyword DAMP to be followed by the input value of parameter
$xdamp$, which denotes the fractional amount of damping used in the
formula explained below. \medskip{}
\\
$F^{(i)}=xdamp\: F^{(i)}+(1-xdamp)\: F^{(i-1)}$\medskip{}
\\
where $F^{(i)}$ is the Fock matrix in the i-th iteration, where
$0\leqq xdamp\leqq1$. \medskip{}
\\
For example, if one intends to use $50\%$ mixing the input will
read:\\
\\
\#DAMPING WILL BE USED\\
DAMP\\
0.5\\
\\
if in case no Fock matrix mixing is required, then the input file
reads,\medskip{}
\\
\# NO DAMPING TO BE USED\\
NODAMP \\

\item Next input consists of data related to the hopping matrix elements.
The input differs slightly for finite systems as compared to the case
when band structure for an infinite polymer is desired within the
H�ckel model. For finite systems one has the option of explicitly
providing the hopping matrix elements or for their automatic generation
by attaching a given hopping value with a certain bond distance. Below
we provide examples to illustrate all the three cases. \\
\medskip{}
\\
First we present the example input associated with the band structure
calculation of the polymer PPP, within the H�ckel model (\emph{cf.}
example file ppp\_band.dat)\\
\medskip{}
\\
\#Read the no. of unique intracell and intercell hopping matrix
elements\\
1 1 \#number of unique intracell and intercell hoppings\\
-2.4,6 \# value of intracell hopping and number of site pairs connected
by it \\
2,1 \\
3,1 \\
4,2 \\
5,3 \\
6,4 \\
6,5 \\
-2.23,1 \# value of intercell hopping and its connectivity \\
6,1,1 \# site pair which is connected by it, followed by the primitive
vector of the cell\\
\medskip{}
\\
Next we present the example of a bond alternated polyene oligomer,
made up of identical double bonds (\emph{cf.} example file tpa10.dat
or tpa11.dat).\\
\\
\#HOPPING PARAMETERS \\
\#Number of unique hoppings \\
2 \\
-2.568,1 \# intracell double bond hopping and number of site pairs
connected by it\\
2,1 \#intracell sites connected by above hopping \\
-2.232,1 \# intercell single bond hopping and number of site pairs
connected by it \\
3,2 \#intercell sites connected by above hopping\\
\medskip{}
\\
In the next example we illustrate the automatic hopping generation
option through the keyword HOPGEN, meant for finite systems which
can simplify this task tremendously (\emph{cf.} example file trigonal\_zigzag\_benzo6\_uhf.dat).\\
\\
\#HOPPING PARAMETERS \\
\#HOPGEN \\
1 \#no. of unique hopping values \\
1.40 \#bond length corresponding to the value of hopping \\
-2.40 \#value of hopping\\
\\

\item The final card specifies whether or not we want to provide nonzero
values for site energies. In case we want to provide a nonzero value,
it is done through the keyword SITE, explained as below:\\
\medskip{}
\\
\#Site energies to be read\\
SITE\\
4 \#Number of nonzero site energies/cell\\
3, 0.1 \#site 3 is assigned the energy 0.1\\
4, 0.3 \#site 4 is assigned the energy 0.3\\
5, -0.2 \#site 5 is assigned the energy -0.2\\
6, -0.1 \#site 6 is assigned the energy -0.1\\
\medskip{}
\\
However, if no nonzero site energies are to be assigned the following
data suffices: \\
\medskip{}
\\
\#Whether site energies are to be read\\
NO SITE
\end{enumerate}

\end{document}
